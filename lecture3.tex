\section*{Lecture 3の回答}

\subsection*{対数関数と総乗}

$\ln \prod_{n=1}^N x_n^2 $を計算する。$\ln (a \cdot b) = \ln a + \ln b$という性質を総乗の形で一般化すると、

\begin{equation*}
	\ln \prod_{n=1}^N a_n = \sum_{n=1}^N \ln a_n
\end{equation*}

となるので、$\ln \prod_{n=1}^N x_n^2 = \sum_{n=1}^N \ln x_n^2 = \sum_{n=1}^N  2\ln x_n $となる。


\subsection*{対数関数とargmax}

実関数$f(x)$において、$\mathrm{arg}\!\max_{x} f(x) $と $\mathrm{arg}\!\max_{x} \ln f(x) $ の値が同じになることを示す。

実関数$f(x)$が$x=a$において最大になるとする。つまり、$a$の任意の前後の値、$b<a<c$において、$f(b) < f(a)$、$f(c) < f(a)$となる。さて、対数関数は単調増加性を持つので、$x < y \Rightarrow \log x < \log y$となる。これらを組み合わせると、$f(b) < f(a) \Rightarrow \ln f(b) < \ln f(a)$、$f(c) < f(a) \Rightarrow \ln f(c) < \ln f(a)$となる。つまり、対数関数が適用されても、$f(x)$を最大化する値(ここでは$a$)は変わらない、ということである。よって、$\mathrm{arg}\!\max_{x} f(x) $ = $\mathrm{arg}\!\max_{x} \ln f(x) $ が示された。

\subsection*{対数関数とガウス分布}

$N(x | \mu , \sigma^2 ) = \frac{1}{(2\pi \sigma^2)^{1/2}} \exp \bigg( -\frac{1}{2\sigma^2}(x-\mu)^2 \bigg)$に対して対数関数を適用する。

\begin{dmath*}
\ln N(x | \mu, \sigma^2) \\
= \ln \bigg( \frac{1}{(2\pi \sigma^2)^{1/2}} \exp \big( -\frac{1}{2\sigma^2}(x-\mu)^2 \big) \bigg) \\
= \ln \frac{1}{(2\pi \sigma^2)^{1/2}} + \ln \exp \big( -\frac{1}{2\sigma^2}(x-\mu)^2 \big) \quad (\because \log_c (a \cdot b) = \log_c a + \log_c b) \\
= \ln (2\pi \sigma^2)^{-1/2} + \ln \exp \big( -\frac{1}{2\sigma^2}(x-\mu)^2 \big) \\
= -\frac{1}{2} \ln (2\pi \sigma^2) + \ln \exp \big( -\frac{1}{2\sigma^2}(x-\mu)^2 \big) \\
= -\frac{1}{2} \ln 2\pi -\frac{1}{2} \ln \sigma^2  + \ln \exp \big( -\frac{1}{2\sigma^2}(x-\mu)^2 \big) \\
= -\frac{1}{2} \ln 2\pi -\frac{1}{2} \ln \sigma^2  -\frac{1}{2\sigma^2}(x-\mu)^2 \quad (\because  \ln \exp x = \ln e^x = x )  \\
\end{dmath*}


\subsection*{最尤推定}
\begin{equation*}
	L(\mu, \sigma^2) = -\frac{N}{2} \ln 2\pi - \frac{N}{2} \ln \sigma^2  -  \frac{1}{2\sigma^2} \sum_{i=1}^N (x_i-\mu)^2
\end{equation*}

とし、$\frac{\partial L}{\partial \mu } = 0, \frac{\partial L}{\partial \sigma^2 } = 0$を解くと、

\begin{equation*}
	\mu = \frac{1}{N} \sum_{i=1}^N x_i, \quad \sigma^2 = \frac{1}{N} \sum_{i=1}^N (x_i - \mu)^2
\end{equation*}

となることを示す。

\begin{dmath*}
	\frac{\partial L}{\partial \mu } \\
	= \frac{d}{d \mu} \bigg(  -  \frac{1}{2\sigma^2} \sum_{i=1}^N (x_i-\mu)^2 \bigg) \\
	=  -  \frac{1}{2\sigma^2} \sum_{i=1}^N   \frac{d (x_i-\mu)^2}{d \mu} \quad (\because \text{微分の線形性}) \\
	= -  \frac{1}{2\sigma^2} \sum_{i=1}^N   -2(x_i - \mu) \\
	= \frac{1}{\sigma^2} \sum_{i=1}^N x_i - \mu
\end{dmath*}

$\frac{\partial L}{\partial \mu } = 0$を解くと、

\begin{dmath*}
	\frac{1}{\sigma^2} \sum_{i=1}^N x_i - \mu \hiderel{=} 0 \\
	\Leftrightarrow\sum_{i=1}^N x_i\hiderel{=}\sum_{i=1}^N \mu \\
	\Leftrightarrow\sum_{i=1}^N x_i\hiderel{=}N \mu \\
	\Leftrightarrow \mu \hiderel{=} \frac{1}{N} \sum_{i=1}^N x_i \\
\end{dmath*}

同様にして、$\frac{\partial L}{\partial \sigma^2 }$を解く。

\begin{dmath*}
	\frac{\partial L}{\partial \sigma^2 } \\
	= \frac{d}{d \sigma^2} \big( - \frac{N}{2} \ln \sigma^2 \big) + \frac{d}{d \sigma^2} \big(  -  \frac{1}{2\sigma^2} \sum_{i=1}^N (x_i-\mu)^2 \big) \\
	= - \frac{N}{2} \frac{1}{\sigma^2} +  \frac{d}{d \sigma^2} \big(  -  \frac{1}{2\sigma^2} \sum_{i=1}^N (x_i-\mu)^2 \big) \quad (\because \frac{d \ln x}{d x} \hiderel{=} \frac{1}{x}) \\
	=  - \frac{N}{2} \frac{1}{\sigma^2} -  \frac{1}{2} \sum_{i=1}^N (x_i-\mu)^2 \frac{d}{d \sigma^2}  \frac{1}{\sigma^2} \quad (\because \text{微分の線形性}) \\
	=  - \frac{N}{2} \frac{1}{\sigma^2} + \frac{1}{2}\frac{1}{\sigma^4} \sum_{i=1}^N (x_i-\mu)^2
\end{dmath*}

$\frac{\partial L}{\partial \sigma^2 } = 0$を解く。

\begin{dmath*}
	- \frac{N}{2} \frac{1}{\sigma^2} + \frac{1}{2}\frac{1}{\sigma^4} \sum_{i=1}^N (x_i-\mu)^2 \hiderel{=} 0 \\
	\Leftrightarrow \frac{N}{2} \frac{1}{\sigma^2} \hiderel{=} \frac{1}{2}\frac{1}{\sigma^4} \sum_{i=1}^N (x_i-\mu)^2 \\
	\Leftrightarrow N \sigma^2 \hiderel{=} \sum_{i=1}^N (x_i-\mu)^2 \\
	\Leftrightarrow \sigma^2 \hiderel{=} \frac{1}{N} \sum_{i=1}^N (x_i-\mu)^2 \\
\end{dmath*}


\subsection*{$\beta$についての最尤推定}

$\ln p(\mathbf{t} | \mathbf{x}, \mathbf{w}, \beta) = -\frac{N}{2} \ln (2\pi) + \frac{N}{2} \ln \beta  - \frac{\beta}{2} \sum_{n=1}^N  (t_n - y)^2 $とし、$\frac{\partial  \ln p(\mathbf{t} | \mathbf{x}, \mathbf{w}, \beta) }{\partial \beta }(\beta^\star) = 0
$となる$\beta^\star$を求める。

\begin{dmath*}
	\frac{\partial  \ln p(\mathbf{t} | \mathbf{x}, \mathbf{w}, \beta) }{\partial \beta }(\beta^\star) \\
	= \frac{\partial}{\partial \beta}\bigg( \frac{N}{2} \ln \beta - \frac{\beta}{2} \sum_{n=1}^N  (t_n - y)^2   \bigg) \\
	= \frac{N}{2 \beta}  - \frac{1}{2} \sum_{n=1}^N  (t_n - y)^2  \\
\end{dmath*}

$\frac{\partial  \ln p(\mathbf{t} | \mathbf{x}, \mathbf{w}, \beta) }{\partial \beta } = 0$を解くと、

\begin{dmath*}
	\frac{N}{2 \beta}  - \frac{1}{2} \sum_{n=1}^N  (t_n - y)^2 \hiderel{=} 0 \\
	\Leftrightarrow  \frac{N}{2 \beta} \hiderel{=}  \frac{1}{2} \sum_{n=1}^N  (t_n - y)^2 \\
	\Leftrightarrow \frac{1}{\beta} \hiderel{=}  \frac{1}{N} \sum_{n=1}^N  (t_n - y)^2 \\
\end{dmath*} 

