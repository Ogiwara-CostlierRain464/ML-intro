\section*{Lecture 4の回答}

\subsection*{スカラ関数のベクトル微分}

$y$が$\mathbf{w}$の関数のとき、$\frac{\partial}{ \partial \mathbf{w}} y^2 = 2 \bigg( \frac{\partial y }{ \partial \mathbf{w} } \bigg)y$となることを示す。ただし、$\mathbf{w} = (w_1, w_2)^T$とする。

左辺は、
\begin{equation*}
\frac{\partial}{\partial \mathbf{w}} y^2 = \begin{pmatrix}
	\frac{\partial y^2}{\partial w_1} \\
	\frac{\partial y^2}{\partial w_2}   
\end{pmatrix} = \begin{pmatrix}
	2y\frac{\partial y}{\partial w_1} \\
	2y\frac{\partial y}{\partial w_2}   
\end{pmatrix}		
\end{equation*}

右辺は、

\begin{equation*}
2 \bigg( \frac{\partial y }{ \partial \mathbf{w} } \bigg)y =  
2\begin{pmatrix}
	\frac{\partial y}{\partial w_1} \\
	\frac{\partial y}{\partial w_2}   
\end{pmatrix}y = \begin{pmatrix}
	2y\frac{\partial y}{\partial w_1} \\
	2y\frac{\partial y}{\partial w_2}   
\end{pmatrix}		
\end{equation*}


よって、両辺は等しい。同じようにして、$\mathbf{w}$が$N$次元ベクトル場合の場合も示せる。